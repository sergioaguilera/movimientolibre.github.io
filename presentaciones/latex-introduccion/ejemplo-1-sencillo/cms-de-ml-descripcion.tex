%
% Ejemplo sencillo: Un artículo con portada
%

\documentclass[letterpaper]{article}  % Articulo en papel tamaño carta, agregue titlepage para separar la portada

\usepackage[spanish]{babel}           % Idioma español
\usepackage[utf-8]{inputenc}          % Estos archivos de texto son UTF-8

\title{CMS de Movimiento Libre 1.2}   % Título
\author{Guillermo Valdez}             % Autor
\date{2007-08-12}                     % Fecha

\begin{document}  % Inicia el documento

\maketitle        % La portada

\section{A cerca de este CMS}

En esta nueva entrega del \textbf{CMS de Movimiento Libre} he enriquecido un poco más a este programa. Le recuerdo que el sitio web movimientolibre.com es mantenido con este programa y que es software libre bajo la licencia GPL.\par

El objetivo de este CMS es elaborar el contenido de un sitio web a partir de archivos simples, dándoles una buena presentación y organización. No depende de ninguna base de datos. Es un buen ejercicio para aprender el lenguaje \textbf{Ruby}.\par

\section{Lo nuevo de esta versión}

\begin{description}
	\item[La clase Plantilla tiene más cualidades]
		Recibe información general del sitio para preparar la estructura constante de todas las páginas del sitio.
		Puede albergar más de un archivo CSS, esto es para disponer de un archivo CSS para la pantalla y otro para la impresión.
	\item[Mejoras en el menú de la izquierda]
		Las opciones del menú cambian de color, controlado por el archivo CSS.
		La anterior opción se dehabilita cuando son gráficos los que hacen el vínculo.
	\item[Configuración en un solo archivo]
		El script ejecutable alberga la mayoría de las configuraciones del sitio.
\end{description}

\section{Requerimientos}

\begin{itemize}
	\item GNU/Linux.
	\item Ruby 1.8 o mejor.
	\item La gema RedCloth.
\end{itemize}

\section{Forma de usarlo}

\begin{itemize}
	\item Desempaque el archivo .tar.gz
	\item Edite la configuración en el script movimientolibre.rb
	\item Ejecute ./movimientolibre.rb y vea el resultado en su navegador.
	\item Modifique el contenido, vea los ejemplos de los directorios articulos, contacto, licencias y manuales. Lo más fácil es duplicar un archivo de contenido y editarlo.
	\item Elabore los menús modificando el contenido del directorio menus.
	\item Modifique el diseño del sitio web alterando el archivo CSS.
	\item Ejecute ./movimientolibre.rb después de cambiar los archivos .rb
\end{itemize}

\end{document}   % Fin del documento

